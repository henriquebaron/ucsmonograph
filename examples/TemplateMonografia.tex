\documentclass[12pt,oneside,english,brazil,lmodern]{ucsmonograph}
\usepackage[brazil]{babel}
\usepackage[utf8]{inputenc}
\usepackage[T1]{fontenc}
\usepackage{graphicx}
\usepackage{pdfpages}
\usepackage{amsmath}
\usepackage{glossaries-extra} % Desnecessário a partir da v1.0.1

\titulo{Meu trabalho de conclusão de curso}
\autor{Pedro Silva}
\data{2018}
\instituicao{Universidade de Caxias do Sul}
\local{Caxias do Sul}
\preambulo{Trabalho de conclusão de curso apresentado à Universidade de Caxias do Sul como requisito parcial à obtenção de grau de Engenheiro Mecânico. Área de concentração: Projetos de Máquinas: Estática e Dinâmica Aplicada.}
\orientador{Dr. Orientador do Meu Trabalho}

% Configuração do pacote hyperref, segundo manual do abnTeX2: não remover
\makeatletter
\hypersetup{%
	unicode=true,
	pdftitle={\@title},
	pdfauthor={\@author},
	pdfsubject={\imprimirpreambulo},
	pdfkeywords={Minhas. Palavras. Chave}
	pdfcreator={LaTeX with ucsmonograph},
	colorlinks=true,
	linkcolor=black,
	citecolor=black,
}
\makeatother

\begin{document}
	\setlength{\afterchapskip}{20pt} % Desnecessário a partir da v1.0.1
	\imprimircapa
	\imprimirfolhaderosto
	
	\imprimirfolhadeaprovacao[Empresa ABC Ltda.]{10/12/2018}{Dr. Avaliador Banca 1}{Me. Avaliador Banca 2}{Eng. Avaliador Externo}
	
	\begin{dedicatoria}
		\vspace*{\fill}
		\hspace{.45\textwidth}
		\begin{minipage}[b]{.5\textwidth}
			\SingleSpacing%
			Dedico este trabalho à minha mãe, meu pai e meu cachorro\dots
		\end{minipage}
	\end{dedicatoria}

	\begin{resumo}
		\SingleSpacing
		Este é o resumo do trabalho.
		\vspace{\onelineskip}
		
		\noindent
		\textbf{Palavras-chave}: Palavras. Chave. Deste. Trabalho.
	\end{resumo}
	
	\begin{resumo}[ABSTRACT]
		\SingleSpacing
		This is the abstract of this work.
		\vspace{\onelineskip}
		
		\noindent
		\textbf{Keywords}: Keywords. Of. This. Work.
	\end{resumo}
	
	\pdfbookmark[0]{\listfigurename}{lof}
	\listoffigures*
	\cleardoublepage
		
	\pdfbookmark[0]{\listquadroname}{loq}
	\listofquadros*
	\cleardoublepage
	
	\pdfbookmark[0]{\listtablename}{lot}
	\listoftables*
	\cleardoublepage
	
	\pdfbookmark[0]{\listadesiglasname}{loa}
	\begin{siglas} % Não esquecer de ordenar alfabeticamente!
		\item[ABNT] Associação Brasileira de Normas Técnicas
		\item[UCS] Universidade de Caxias do Sul
		\item[XML] \foreignlanguage{english}{Extensible Markup Language}
	\end{siglas}

	\pdfbookmark[0]{\listadesimbolosname}{los}
	\begin{simbolos}
		\item[\ensuremath{\alpha}] Ângulo qualquer [rad]
		\item[\ensuremath{\tau}] Período [s]
		\item[\ensuremath{\omega}] Velocidade angular [rad/s]
		\item[\ensuremath{t}] Tempo [s]
	\end{simbolos}

	\pdfbookmark[0]{\contentsname}{toc}
	\tableofcontents*
	
	\textual % Início dos elementos textuais
	
	\chapter{Introdução}
	Introdução do trabalho
	
	\section{Justificativa}
	Justifique a proposta.
	
	\section{Ambiente de desenvolvimento}
	Este trabalho foi desenvolvido na empresa A.
	
	\section{Objetivos}
	Aqui ficam os objetivos do trabalho.
	
	\subsection{Objetivos gerais}
	Objetivos gerais\dots
	
	\subsection{Objetivos específicos}
	Objetivos específicos\dots
	
	\chapter{Referencial teórico}
	Aqui ficam os fundamentos teóricos do trabalho proposto \cite{rao:2008}.
	
	\section{Método de Rayleigh-Litz}
	O método de Rayleigh-Litz \cite{cooley:1965}.
	
	\postextual
	
	\bibliography{Bibliografia}
	
\end{document}